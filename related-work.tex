User interface structure has also been the focus of prior
work~\cite{Azim2013, yang-ase15,yang2013grey}. GATOR~\cite{yang-ase15}
is a toolkit for Android that perform static reference analysis for GUI
objects and control-flow analysis for GUI related event-callbacks to
obtain GUI layout hierarchy and GUI transition graph for test
generation. A\textsuperscript{3}E~\cite{Azim2013} leverages static
taint-analysis to capture and model transitions between app's activity
from app's bytecode without modifying OS system and app's source code to
explore Android app's activities and to achieve high coverage. Their
work focus on analyzing event-callbacks and generating event sequences
without generating data for input widgets on a GUI screen, which normally
has low coverage for real-world application. 

There are many great research works that leverage NLP technique to assess security risk and 
identify malicious behaviors~\cite{gtgz:14,hlxw:15,Huang:2014,nmyz:15,pxy:13}.  
WHYPER~\cite{pxy:13} and Chabada~\cite{gtgz:14} both apply NLP technique on apps' 
description to abstract user's expectation, and further evaluate the coherence between 
expectation and implementation. In particular, WHYPER examines whether sensitive 
permissions declared by the apps are needed regarding apps' description. Chabada examines 
apps' API usage regarding their descriptions. AsDroid check the coherence between text of on 
the GUI associated with event callbacks and its triggered sensitive behavior. Our technique also 
complement their works by providing complete contextual information of GUI to further achieve 
context awareness to protect user's privacy.  